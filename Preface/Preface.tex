% $Author$
% $Date$
% $Revision$

% HISTORY:
% 2006-10-05 - Oscar started
% 2007-05-28 - Stef edit
% 2007-06-06 - Oscar first draft
% 2007-08-14 - Stef corrections
% 2007-09-06 - Lukas review
% 2009-08-12 - Oscar rewrite for Pharo

%=================================================================
\ifx\wholebook\relax\else
% --------------------------------------------
% Lulu:
	\documentclass[a4paper,10pt,twoside]{book}
	\usepackage[
		papersize={6.13in,9.21in},
		hmargin={.75in,.75in},
		vmargin={.75in,1in},
		ignoreheadfoot
	]{geometry}
	\input{../common.tex}
	\pagestyle{headings}
	\setboolean{lulu}{true}
% --------------------------------------------
% A4:
%	\documentclass[a4paper,11pt,twoside]{book}
%	\input{../common.tex}
%	\usepackage{a4wide}
% --------------------------------------------
    \graphicspath{{figures/} {../figures/}}
	\begin{document}
	% \renewcommand{\nnbb}[2]{} % Disable editorial comments
	\sloppy
	\frontmatter
\fi
%=================================================================
\chapter{Prefácio}\chalabel{intro}

%=================================================================
\section*{O que é \pharo?}

\pharo é uma completa e moderna implementação de código aberto da linguagem de programação e do ambiente \st. \pharo se derivou do projeto \squeak\cite{Inga97a}, uma re-implementação do clássico sistema \st-80. Ao passo que \squeak foi desenvolvido principalmente como uma plataforma para desenvolvimento de software educacional experimental, \pharo esforça-se em oferecer uma plataforma leve e de código aberto para desenvolvimento profissional de software, além de uma plataforma robusta e estável para pesquisa e desenvolvimento com linguagens e ambientes dinâmicos. \pharo serve como a implementação referência para o framework web Seaside.

\pharo resolve alguns problemas de licenciamento que existiam no \squeak. Diferentemente das versões anteriores do \squeak, \pharo contém somente código licenciado sob a licença MIT. O projeto \pharo teve seu início em Março de 2008 como um fork do \squeak 3.9, e sua primeira versão 1.0 beta foi lançada em 31 de Julho de 2009.

Embora tenha removido vários pacotes existentes no \squeak, \pharo inclui várias funcionalidades que são opcionais no \squeak. Por exemplo, fontes true type foram incluídas no \pharo. \pharo também inclui suporte real a true block closures. As interfaces gráficas do usuário também foram simplificadas e revisadas.

\pharo é altamente portável --- até mesmo sua máquina virtual é escrita inteiramente em \st, tornando-o fácil de debugar, analisar, e modificar. \pharo é o veículo condutor de uma ampla gama de projetos inovadores que vão desde de aplicações multimídia e plataformas educacionais até ambientes comerciais de desenvolvimento web. \diogenes{Pharo is the vehicle for a wide range of innovative projects...}

Há um aspecto importante por trás do projeto \pharo: \pharo não será uma cópia do passado, mas realmente uma \emph{reinvenção} do \st. Abordagens do tipo "Big-bang", (isto é, onde tudo muda de uma só vez), raramente são bem sucedidas. \pharo irá favorecer mudanças incrementais e evolucionárias. Queremos ser capazes de experimentar novas funcionalidades e bibliotecas importantes. Evolução significa que \pharo aceita os erros e não tem como objetivo a próxima solução perfeita dando um grande passo apenas\,---\, mesmo que amássemos isso. \pharo favorecerá pequenas mudanças incrementais, mas em grande escala. O sucesso de \pharo depende das contribuições de sua comunidade.
% The \pharo community will pay attention to your submissions to improve the system.

%=================================================================
\section*{Quem deve ler esse livro?}

Esse livro é baseado no livro \emph{Squeak by Example}\footnote{\sbe}, que é uma introdução open-source ao \squeak.
O livro foi livremente adaptado e revisado para refletir as diferenças entre \pharo e \squeak.
Esse livro apresenta vários aspectos do \pharo, começando com o básico, e depois indo para tópicos mais avançados.

Esse livro não ensinará você a programar. O leitor já deve possuir alguma familiaridade com linguagens de programação. Alguma experiência com programação orientada a objetos será bem vinda.

Esse livro irá apresentar o ambiente de programação \pharo, a linguagem e as ferramentas relacionadas.  Você será exposto a idiomas e práticas comuns, mas o foco é na tecnologia, não no design orientado a objetos. Onde for possível, nós iremos mostrar a você vários exemplos. (Fomos inspirados pelo excelente livro de Alec Sharp sobre Smalltalk\cite{Shar97a}.)
\index{Sharp, Alex}

Há vários outros livros gratuitos sobre \st na web, mas nenhum deles focaliza-se especificamente no ambiente \pharo. Veja alguns destes livros em:
\url{http://stephane.ducasse.free.fr/FreeBooks.html}

\ifluluelse{}{\newpage} % layout hint
%=================================================================
\section*{Uma palavra de cautela}

% http://www.surfscranton.com/architecture/KnightsPrinciples.htm

Não fique frustrado pelas partes do \st que você não entender imediatamente.
Você não precisa saber tudo!
Alan Knight expressa esse princípio da seguinte forma \footnote{\url{http://www.surfscranton.com/architecture/KnightsPrinciples.htm}}:
\index{Knight, Alan}
\important{{\bf Tente não se importar.}
Programadores iniciantes em \st frequentemente tem problemas porque pensam que precisam entender todos os detalhes de como uma coisa funciona antes que possam usá-la. Isso significa que leva um bom tempo até que tenham dominado o \ct{Transcript show: 'Hello World'}. Um dos grandes saltos em OO é ser capaz de responder a pergunta ``Como isso funciona?'' com ``Não me importa''.}

%=================================================================
\section*{An open book}

This book is an open book in the following senses: 

\begin{itemize}

\item	The content of this book is released under the Creative Commons Attribution-ShareAlike (by-sa) license.
		In short, you are allowed to freely share and adapt this book, as long as you respect the conditions of the license available at the following URL: 
		\url{http://creativecommons.org/licenses/by-sa/3.0/}.

\item	This book just describes the core of \pharo.
		Ideally we would like to encourage others to contribute chapters
		on the parts of \pharo that we have not described.
		If you would like to participate in this effort, please
		contact us.  We would like to see this book grow!
\end{itemize}

For more details, visit \pbe.

%=================================================================
\section*{The \pharo community}

The \pharo community is friendly and active.
Here is a short list of resources that you may find useful:

\begin{itemize}
\item \url{http://www.pharo-project.org} is the main web site of \pharo.
%environment built on top of \pharo but whose audience is elementary
%school teachers.) % I remove this [Martial: french contributor]

\item \url{http://www.squeaksource.com} is the equivalent of SourceForge for \pharo projects.
Many optional packages for \pharo live here.
\end{itemize}

%=================================================================
\section*{Examples and exercises}

We make use of two special conventions in this book.

We have tried to provide as many examples as possible.
In particular, there are many examples that show a fragment of code which can be evaluated.  We use the symbol \ct{-->} to indicate the result that you obtain when you select an expression and \menu{print it}:

\begin{code}{@TEST}
3 + 4 --> 7    "if you select 3+4 and 'print it', you will see 7"
\end{code}

In case you want to play in \pharo with these code snippets, you can download a plain text file with all the example code from the book's web site: \pbe.

The second convention that we use is to display the icon \dothisicon{} to indicate when there is something for you to do:

\dothis{Go ahead and read the next chapter!}

%=================================================================
\section*{Acknowledgments}

We would first like to thank the original developers of \squeak for making this amazing \st development environment available as an open source project.

% We would like to thank various people who have contributed to this book.
% In particular, we thank
We would also like to thank Hilaire Fernandes and Serge Stinckwich who allowed us to translate parts of their columns on \st, and Damien Cassou for contributing the chapter on streams.

We especially thank Alexandre Bergel, Orla Greevy, Fabrizio Perin, Lukas Renggli, Jorge Ressia and Erwann Wernli for their detailed reviews.

We thank the University of Bern, Switzerland, for graciously supporting this open-source project and for hosting the web site of this book.

We also thank the Squeak community for their enthusiastic support of this book project, and for informing us of the errors found in the first edition of this book.

%=============================================================
\ifx\wholebook\relax\else
   \bibliographystyle{jurabib}
   \nobibliography{scg}
   \end{document}
\fi
%=============================================================
